% performance tools 2018-2019
\documentclass[a4paper,pagesize,12pt]{scrbook}

\usepackage{verbatim} % for comments
\usepackage{listings} 
\usepackage{xcolor}
\definecolor{mycolor}{rgb}{0.86666,0.78431,0.78431}
\usepackage{url}
\usepackage{hyperref}
% \usepackage{blindtext}
\newcommand{\q}[1]{>>\textit{#1}<<}
%\title{A book title}   
%\author{Author Name} 
%\date{\today} 
\usepackage{mwe} 
\usepackage{subfig}

%      #    #    #   ####
%      #    ##  ##  #    #
%      #    # ## #  #
%      #    #    #  #  ###
%      #    #    #  #    #
%      #    #    #   ####
\usepackage{graphicx}
\usepackage[font=small,labelfont=bf]{caption}
% ----- basic figure frame:
\newcommand{\cscsfigure}[4]{
\begin{figure}
\centering
\includegraphics[width=#1\textwidth]{#2}
\captionof{figure}{#3}
\label{#4}
\end{figure}
}

%   #          #     ####    #####     #    #    #   ####
%   #          #    #          #       #    ##   #  #    #
%   #          #     ####      #       #    # #  #  #
%   #          #         #     #       #    #  # #  #  ###
%   #          #    #    #     #       #    #   ##  #    #
%   ######     #     ####      #       #    #    #   ####
\usepackage{listings}
% CSCS color definition
\definecolor{cscsred}{RGB}{172,27,38}
\definecolor{cscsgrey}{RGB}{64,64,64}
\definecolor{cscslgrey}{RGB}{224,224,224}
\definecolor{cscsgreen}{RGB}{114,121,28}
\definecolor{cscsblue}{RGB}{0,122,150}
\definecolor{cscsbrown}{RGB}{151,72,6}
\definecolor{cscspurple}{RGB}{128,0,128}
\definecolor{cscsyellow}{RGB}{167,135,32}
\definecolor{snow}{RGB}{255,250,250}
\definecolor{mintcream}{RGB}{245,255,250}
\definecolor{cscsblack}{RGB}{0,0,0}
\definecolor{cscswhite}{RGB}{255,255,255}
% --------------------------------------------------------------------------
% https://tex.stackexchange.com/questions/83882/how-to-highlight-python-syntax-in-latex-listings-lstinputlistings-command
% --------------------------------------------------------------------------
% Default fixed font does not support bold face
\DeclareFixedFont{\ttb}{T1}{txtt}{bx}{n}{12} % for bold
\DeclareFixedFont{\ttm}{T1}{txtt}{m}{n}{12}  % for normal

% Custom colors
\usepackage{color}
\definecolor{deepblue}{rgb}{0,0,0.5}
\definecolor{deepred}{rgb}{0.6,0,0}
\definecolor{deepgreen}{rgb}{0,0.5,0}

\usepackage{listings}

%   #####    #   #   #####  #    #   ####   #    #
%   #    #    # #      #    #    #  #    #  ##   #
%   #    #     #       #    ######  #    #  # #  #
%   #####      #       #    #    #  #    #  #  # #
%   #          #       #    #    #  #    #  #   ##
%   #          #       #    #    #   ####   #    #
% Python style for highlighting
\newcommand\pythonstyle{\lstset{
language=Python,
basicstyle=\ttm,
otherkeywords={self},             % Add keywords here
keywordstyle=\ttb\color{deepblue},
emph={MyClass,__init__},          % Custom highlighting
emphstyle=\ttb\color{deepred},    % Custom highlighting style
stringstyle=\color{deepgreen},
frame=tb,                         % Any extra options here
showstringspaces=false,
backgroundcolor=\color{mintcream}
%backgroundcolor=\color{cscslgrey}
}}


% Python environment
\lstnewenvironment{python}[1][]
{
\pythonstyle
\lstset{#1}
}
{}
% --------------------------------------------------------------------------
%   #####      #    #####
%   #    #     #    #    #
%   #####      #    #####
%   #    #     #    #    #
%   #    #     #    #    #
%   #####      #    #####
% \usepackage[style=ieee,backend=bibtex,minnames=2,maxnames=3]{biblatex}
% HOWTO: pdflatex 0.tex;biber 0;pdflatex 0.tex # pdfTeX 3.14159265-2.6-1.40.19 (TeX Live 2018)
\usepackage[backend=biber]{biblatex}
\addbibresource{0.bib}

\begin{document}

\tableofcontents

\chapter*{Python performance tools}

%{{{ Intro
\section{Introduction}

This document gives an overview of the performance tools able to collect
performance data of parallel Python applications.

For testing purpose, the \href{https://pysph.readthedocs.io/en/latest/}{PySPH}
scientific code will be used. PySPH is an open source framework for Smoothed Particle
Hydrodynamics (SPH) simulations. It is implemented in Python/Cython for
high-performance and can be run in parallel using OpenMP and/or MPI (mpi4py)
together with Trilinos/Zoltan.

A simple test case has been used to benchmark PySPH and the performance tools
on Piz Daint.  This test case creates a cube shaped block of water falling in
free-space under the influence of gravity while solving the incompressible,
inviscid flow equations of motion. The performance data used in this document
is archived in \underline{/apps/daint/UES/6.0.UP07/python/}. Strong scaling
results will be described in the next section. 

%}}}

%{{{ Results
\section{Scaling results}
%\section*{Scaling results}

The (cpu only) simulation of the cube with PySPH shows good strong scaling up
to 6 nodes on daint-gpu with a total number of $10^7$ particles (Fig.
\ref{strongxnodes}). The best timings are reached with 12 mpi tasks and 1
openmp thread per compute node ((Fig. \ref{strong1node})).

\cscsfigure{0.5}{img/scaling/strong_xnodes.png}{Strong scaling results of the PySPH
cube simulation between 1 and 6 daint-gpu (cpu only) compute nodes
(\texttt{-{}-}max-steps 1000 \texttt{-{}-}disable-output \texttt{-{}-}np
$10^7$) running with 12 mpi tasks per node. The Haswell cpu holds a maximum of
12 cores per compute node). Parallel efficiency drops below 50\% above 6 compute
nodes.}{strongxnodes}

\cscsfigure{0.4}{img/scaling/strong_1node.png}{Strong scaling results of the PySPH cube
simulation on 1 daint-gpu (cpu only) compute node (\texttt{-{}-}max-steps 1000
\texttt{-{}-}disable-output \texttt{-{}-}np $10^7$). The fastest runtime (in
seconds) is reached with 12 mpi tasks times 1 openmp thread.}{strong1node}

% \cscsfigure{1}{img/particles_per_core.png}{Number of particles per core
% relative to the total number of processes (i.e $n_{mpi\_tasks} *
% n_{openmp\_threads}$), for a total number of $10^7$ particles, where the
% Haswell cpu on daint-gpu holds a maximum of 12 cores per compute
% node.}{ptpercore}

%}}} 

\newpage
%{{{ cProfile
%    ####   #####   #####    ####   ###### # #       ######
%   #    #  #    #  #    #  #    #  #      # #       #
%   #       #    #  #    #  #    #  #####  # #       #####
%   #       #####   #####   #    #  #      # #       #
%   #    #  #       #   #   #    #  #      # #       #
%    ####   #       #    #   ####   #      # ######  ######
% \section*{cProfile}
\section{cProfile}

Python's built-in profiler \href{https://docs.python.org/3/library/profile.html}{cProfile}
can be used to collect performance data for single process or multi processes
Python codes (using mpi4py for example). 
While it is possible to collect performance data from a serial program without
modifying the python code (via \textit{srun -n1 python -m cProfile -o
output.prof mycode.py}), it is required to manually instrument a parallel code
with cProfile calls:

%ok: \begin{python}[backgroundcolor=\color{cscslgrey}]
{\footnotesize
\begin{python}
!#/usr/bin/env python3
import cProfile, sys                               # cProfile
from mpi4py import MPI
...

if __name__ == '__main__':    
    cscs_rank = MPI.COMM_WORLD.Get_rank()
    pr = cProfile.Profile()                        # cProfile
    pr.enable()                                    # cProfile

    app = Cube()                         # main function call
    app.run()                            # main function call

    pr.disable()                                   # cProfile

    # binary dump the performance data for each rank with:
    pr.dump_stats('cpu_%d.cprofile' % cscs_rank)   # cProfile
\end{python}
}

You can now run your python code as usual with:
\begin{itemize}
%\scriptsize
%    \item[*] {reframe: lammps\_sph\_strong\_scaling\_100000wp+scalasca\_profiling.py}
%    \item[] i.e \tiny{\textbf{scalasca -analyze} srun lmp\_mpi ...}
    \item[$>$] \textbf{srun -n12 python mycode.py}
%    \item[]
%    \scriptsize
%    \item[*] {\textbf{cube} ./scorep\_lmp\_mpi\_12\_sum/summary.cubex \&}
\end{itemize}

This will generate several performance output files (one for each MPI rank): 
\textbf{cpu\_N.cprofile}. These files can be visualized with
%gprof2dot$^\textrm{\ref{sec:gprof2dot}}$ and or
\hyperref[sec:gprof2dot]{gprof2dot} and or
\hyperref[sec:SnakeViz]{SnakeViz}.
These postprocessing tools will be described in the next section.

% --------------------------------------------------------------------------
\subsection{gprof2dot}
\label{sec:gprof2dot}

\href{https://pypi.org/project/gprof2dot}{gprof2dot} is a Python script to
convert the output from many profilers (including cProfile) into a dot graph.
To generate an image file from the \textbf{cpu\_N.cprofile} file(s), you may run:
%\begin{itemize}
%    \item[$>$] \textbf{gprof2dot -f pstats} cpu\_0.cprofile > cpu\_0.cprofile.dot
%    \item[$>$] \textbf{dot -Tpng} cpu\_0.cprofile.dot -o cpu\_0.cprofile.dot.png
%\end{itemize}
\begin{python}
    module load daint-gpu # or daint-mc
    module load gprof2dot
    gprof2dot -f pstats cpu_0.cprofile > cpu_0.cprofile.dot
    dot -Tpng cpu_0.cprofile.dot -o cpu_0.cprofile.dot.png
\end{python}

and open the resulting png image file with your favorite png editor. Refer to
the \href{https://github.com/jrfonseca/gprof2dot#output}{\textbf{documentation}} for
more details.

\cscsfigure{0.7}{img/cprofile/01cn_cpu_0-cprofile-dot.png}
{cProfile callgraph of the PySPH cube simulation on 1 node (12 cores)}{cprofile01}

\cscsfigure{0.7}{img/cprofile/02cn_cpu_0-cprofile-dot.png}
{cProfile callgraph of the PySPH cube simulation on 2 nodes (24 cores)}{cprofile02}

\cscsfigure{0.7}{img/cprofile/04cn_cpu_0-cprofile-dot.png}
{cProfile callgraph of the PySPH cube simulation on 4 nodes (48 cores)}{cprofile04}

\cscsfigure{0.7}{img/cprofile/06cn_cpu_0-cprofile-dot.png}
{cProfile callgraph of the PySPH cube simulation on 6 nodes (72 cores)}{cprofile06}

% --------------------------------------------------------------------------
\newpage
\subsection{SnakeViz}
\label{sec:SnakeViz}

\href{https://pypi.org/project/snakeviz/}{SnakeViz} is a graphical viewer that
can convert the output from cProfile into an interactive webpage. To generate
the report from the \textbf{cpu\_N.cprofile} file(s), you may run:
\begin{python}
    snakeviz cpu_0.cprofile
\end{python}

This should open an interactive window in your default web browser that
displays the performance data from your .cprofile file. The top functions are 
listed at the bottom of the report (Fig.~\ref{snaketop3}).

\begin{itemize}
\itemsep0em 
\scriptsize
	\item \textbf{ncalls}: Total number of calls to the function.
	\item \textbf{tottime}: Total (exclusive) time spent in the function, not including time spent in calls to sub-functions
	\item \textbf{percall}: tottime divided by ncalls
	\item \textbf{cumtime}: Cumulative (inclusive) time spent in this function and all sub-functions
	\item \textbf{percall}: cumtime divided by ncalls
	\item \textbf{filename}:lineno(function): File name and line number were the function is defined
\end{itemize}

% --------------------------------------------------------------------------
\begin{figure}[ht]
\centering
\caption{SnakeViz view of the top 3 function calls}
\label{snaketop3}
\begin{minipage}[c]{\textwidth}
\captionsetup{labelformat=empty} \centering
    \includegraphics[width=\textwidth]{img/cprofile/snakeviz/01top3.png}
    \captionof{figure}{1 node (12 cores)}
    %\label{snake1cntop3}
\end{minipage}
\noindent
\begin{minipage}[c]{\textwidth}
\captionsetup{labelformat=empty} \centering
    \includegraphics[width=\textwidth]{img/cprofile/snakeviz/02top3.png}
    \captionof{figure}{2 nodes (24 cores)}
    %\label{snake2cntop3}
\end{minipage}
\noindent
\begin{minipage}[c]{\textwidth}
\captionsetup{labelformat=empty} \centering
    \includegraphics[width=\textwidth]{img/cprofile/snakeviz/04top3.png}
    \captionof{figure}{4 nodes (48 cores)}
    %\label{snake4cntop3}
\end{minipage}
\noindent
\begin{minipage}[c]{\textwidth}
\captionsetup{labelformat=empty} \centering
    \includegraphics[width=\textwidth]{img/cprofile/snakeviz/06top3.png}
    \captionof{figure}{6 nodes (72 cores)}
    %\label{snake6cntop3}
\end{minipage}
\end{figure}

Data can be displayed in a \textbf{sunburst} plot (Fig. \ref{snake6cnsunburst}):
SnakeViz displays profiles as a sunburst in which functions are represented as
arcs. A root function is a circle at the middle, with functions it calls
around, then the functions those functions call, and so on. The amount of time
spent inside a function is represented by the angular width of the arc. An arc
that wraps most of the way around the circle represents a function that is
taking up most of the time of its calling function, while a skinny arc
represents a function that is using hardly any time at all.
Clicking on a particular function which will generate a new plot with the
selected function now at the center.

% --------------------------------------------------------------------------
\cscsfigure{0.7}{img/cprofile/snakeviz/06sunburst.png}
{SnakeViz sunburst view}{snake6cnsunburst}

Data can also be displayed in an \textbf{icicle} plot (Fig.~\ref{snake}):
SnakeViz displays profiles in which functions are represented as
squares. A root function is a square at the top, with functions it calls
below, then the functions those functions call, and so on. The amount of time
spent inside a function is represented by the size of the square.

Refer to
the \href{https://jiffyclub.github.io/snakeviz/#interpreting-results}{\textbf{documentation}} for
more details.

% --------------------------------------------------------------------------
\begin{figure}[ht]
\centering
    \caption{SnakeViz view of the calltree}
    \label{snake}
\begin{minipage}[c]{\textwidth}
\captionsetup{labelformat=empty} \centering
    \includegraphics[width=.5\textwidth]{img/cprofile/snakeviz/01.png}
    \captionof{figure}{1 node (12 cores)}
    %\label{snake1cn}
\end{minipage}
\noindent
\begin{minipage}[c]{\textwidth}
\captionsetup{labelformat=empty} \centering
    \includegraphics[width=.5\textwidth]{img/cprofile/snakeviz/02.png}
    \captionof{figure}{2 nodes (24 cores)}
    %\label{snake2cn}
\end{minipage}
\noindent
\begin{minipage}[c]{\textwidth}
\captionsetup{labelformat=empty} \centering
    \includegraphics[width=.5\textwidth]{img/cprofile/snakeviz/04.png}
    \captionof{figure}{4 nodes (48 cores)}
    %\label{snake4cn}
\end{minipage}
\noindent
\begin{minipage}[c]{\textwidth}
\captionsetup{labelformat=empty} \centering
    \includegraphics[width=.7\textwidth]{img/cprofile/snakeviz/06.png}
    \captionof{figure}{6 nodes (72 cores)}
    %\label{snake6cn}
\end{minipage}
\end{figure}


%}}} 

%{{{ scorep

%   ####    ####    ####   #####   ######  #####
%  #       #    #  #    #  #    #  #       #    #
%   ####   #       #    #  #    #  #####   #    #
%       #  #       #    #  #####   #       #####
%  #    #  #    #  #    #  #   #   #       #
%   ####    ####    ####   #    #  ######  #
%\subsection*{http://www.score-p.org}
% --------------------------------------------------------------------------
\newpage
\section{Score-P}
Score-P (Scalable Performance Measurement Infrastructure for Parallel Codes) is
an highly scalable tool suite for profiling, event tracing, and online
analysis of HPC applications developed by \url{https://www.vi-hps.org} institute.
The \href{https://github.com/score-p/scorep_binding_python}{scorep\_binding\_python}
allows the profiling and tracing of Python code:

\begin{python}
    module load daint-gpu # or daint-mc
    module load scorep_binding_python Score-P
    export SCOREP_ENABLE_PROFILING=true          # for profiling
    export SCOREP_ENABLE_TRACING=false           # for profiling
#   export SCOREP_ENABLE_PROFILING=true          # for tracing
#   export SCOREP_ENABLE_TRACING=false           # for tracing
    export SCOREP_PROFILING_MAX_CALLPATH_DEPTH=6
\end{python}

You can run your python code as usual with:
\begin{itemize}
    \item[$>$] srun -n12 python \textbf{-m scorep --mpi} mycode.py
\end{itemize}

\subsection{CubeGUI}
\href{http://apps.fz-juelich.de/scalasca/releases/cube/4.4/docs/guide/html/}{Cube}
is a graphical viewer that can be used to visualize the output from a Score-P
profiling job. The \textbf{.cubex} file(s) can be opened with:
\begin{python}
    cube scorep-20181224_1116_10115260531853858/profile.cubex
\end{python}

The profiling report for the cube testcase is listed on
figures~\ref{scorepP}-\ref{scorepPCube}.

% --------------------------------------------------------------------------
\begin{figure}[!ht]
\captionsetup[subfigure]{labelformat=empty}
\caption{Score-P profiling of the PySPH cube simulation (summary)}
\label{scorepP}
  \subfloat[1 node (12 cores)\label{scorepP01cn}]{%
  \includegraphics[width=0.3\textwidth]{img/scorep/p/01cn_.png}
  }
  \hfill
  \subfloat[2 nodes (24 cores)\label{scorepP02cn}]{%
  \includegraphics[width=0.3\textwidth]{img/scorep/p/02cn_.png}
  }
\end{figure}%
\vspace{-1cm}
\begin{figure}[!ht]
\captionsetup{labelformat=empty}
\captionsetup[subfigure]{labelformat=empty}
% \caption{}
% \label{scorepPCube}
  \subfloat[4 nodes (48 cores)\label{scorepP04cn}]{%
  \includegraphics[width=0.3\textwidth]{img/scorep/p/04cn_.png}
  }
  \hfill
  \subfloat[6 nodes (72 cores)\label{scorepP06cn}]{%
  \includegraphics[width=0.3\textwidth]{img/scorep/p/06cn_.png}
  }
\end{figure}

% --------------------------------------------------------------------------
\begin{figure}[!ht]
\captionsetup[subfigure]{labelformat=empty}
\caption{Score-P profiling of the PySPH cube simulation (CubeGUI)}
\label{scorepPCube}
  \subfloat[1 node (12 cores)\label{scorepP01cnCube}]{%
  \includegraphics[width=0.48\textwidth]{img/scorep/p/01cn.png}
  }
  \hfill
  \subfloat[2 nodes (24 cores)\label{scorepP02cnCube}]{%
  \includegraphics[width=0.48\textwidth]{img/scorep/p/02cn.png}
  }
\end{figure}%
\vspace{-1cm}
\begin{figure}[!ht]
\captionsetup{labelformat=empty}
\captionsetup[subfigure]{labelformat=empty}
% \caption{}
% \label{scorepPCube}
  \subfloat[4 nodes (48 cores)\label{scorepP04cnCube}]{%
  \includegraphics[width=0.48\textwidth]{img/scorep/p/04cn.png}
  }
  \hfill
  \subfloat[6 nodes (72 cores)\label{scorepP06cnCube}]{%
  \includegraphics[width=0.48\textwidth]{img/scorep/p/06cn.png}
  }
\end{figure}

\vspace{2cm}
\subsection{Vampir}
\href{https://www.vi-hps.org/tools/vampir.html}{Vampir}
is a graphical viewer that can be used to visualize the output from a Score-P
tracing job. The \textbf{.otf2} file(s) can be opened with:
\begin{python}
    vampir scorep-20181224_1139_10118768546329892/traces.otf2
\end{python}

The tracing report for the cube testcase can be seen in figure~\ref{scorepTV}.

% --------------------------------------------------------------------------
\begin{figure}[ht]
\centering
\caption{Score-P tracing of the PySPH cube simulation (Vampir)}
\label{scorepTV}
\begin{minipage}[c]{\textwidth}
\captionsetup{labelformat=empty} \centering
    \includegraphics[width=.6\textwidth]{img/scorep/t/01cn.png}
    \captionof{figure}{1 node (12 cores)}
    %\label{scorepTV1cn}
\end{minipage}
\noindent
\begin{minipage}[c]{\textwidth}
\captionsetup{labelformat=empty} \centering
    \includegraphics[width=.6\textwidth]{img/scorep/t/02cn.png}
    \captionof{figure}{2 nodes (24 cores)}
    %\label{scorepTV2cn}
\end{minipage}
\noindent
\begin{minipage}[c]{\textwidth}
\captionsetup{labelformat=empty} \centering
    \includegraphics[width=.6\textwidth]{img/scorep/t/04cn.png}
    \captionof{figure}{4 nodes (48 cores)}
    %\label{scorepTV4cn}
\end{minipage}
\noindent
\begin{minipage}[c]{\textwidth}
\captionsetup{labelformat=empty} \centering
    \includegraphics[width=.6\textwidth]{img/scorep/t/06cn.png}
    \captionof{figure}{6 nodes (72 cores)}
    %\label{scorepTV6cn}
\end{minipage}
\end{figure}

%Refer to
%the \href{}{\textbf{documentation}} for
%more details.

% \begin{itemize}
% \setlength\itemsep{0.1em}
%     \item Cube GUI and plugins by Juelich Supercomputing Centre
%     \item Vampir by Technische Universitat Dresden
% \end{itemize}

%}}}

%{{{ intel vtune

%  #    #   #####  #    #  #    #  ######
%  #    #     #    #    #  ##   #  #
%  #    #     #    #    #  # #  #  #####
%  #    #     #    #    #  #  # #  #
%   #  #      #    #    #  #   ##  #
%    ##       #     ####   #    #  ######
% --------------------------------------------------------------------------
\newpage
\section{Intel Vtune Amplifier}

\href{https://software.intel.com/en-us/vtune-amplifier-help-introduction}{Vtune
Amplifier} is one of Intel performance analysis tool. It can collect
performance data of Python code:

\begin{python}
    module load daint-gpu # or daint-mc
    module swap PrgEnv-cray PrgEnv-intel
    XX=/apps/common/UES/intel/2019
    VV=$XX/vtune_amplifier
    source $VV/amplxe-vars.sh
    amplxe-cl -V
\end{python}

You can run your python code as usual with:
\begin{itemize}
    \item[$>$] chmod u+x mycode.py
    \item[$>$] VVV='amplxe-cl -collect hotspots -trace-mpi -data-limit=0 -r hotspots -search-dir bin:=mysrc'
    \item[$>$] srun -n12 \textbf{\$VVV} mycode.py
\end{itemize}

Some further explanation for why we use these arguments:

\begin{itemize}
\itemsep0em 
\scriptsize
	\item \textbf{-collect hotspots}: use hotspots as collection mode (amplxe-cl -help collect).
	\item \textbf{-trace-mpi}: asks to collect MPI data. If you don't have an MPI application, don't add this flag.
    \item \textbf{-r hotspots}: write results to the specified directory (which doesn't have to exist).
    \item \textbf{-search-dir}: where to look for the Python scripts or binary (call-stack information).
    \item \textbf{-finalization-mode=none}: instructs Vtune to skip the finalization step (not needed on daint).
\end{itemize}

The performance data can be visualized with \textbf{amplxe-gui}:
\begin{python}
    amplxe-gui hotspots.nid00000
\end{python}

%The profiling report for the cube testcase is listed on
%figures~\ref{scorepP}-\ref{scorepPCube}.

The overview tab (Fig.~\ref{vtune_summary}) will display some high level results.
\cscsfigure{0.7}{img/vtune/01cn_summary.png}{Vtune amplifier: summary view (1 node, 12 cores)}{vtune_summary}

For more specific information, try selecting the Bottom-up tab (Fig.~\ref{vtune_bottomup}).
\cscsfigure{0.7}{img/vtune/01cn_bottomup.png}{Vtune amplifier: bottom-up view (1 node, 12 cores)}{vtune_bottomup}

You may have to specify the path to your src code to get detailled information
about your application (Fig.~\ref{vtune_src}).
\cscsfigure{0.7}{img/vtune/01cn_cpp.png}{Vtune amplifier: src path configuration}{vtune_src}

% --------------------------------------------------------------------------
% Vtune has limited functionality for Python applications: only 3 types of
% collection modes are currently supported:
%     hotspots (general performance overview)
%     memory-consumption (memory-specific information)
%     locksandwaits (threading analysis)
% --------------------------------------------------------------------------

%}}}

%{{{ conclusion
\newpage
\section{Conclusion}
Collecting and visualizing performance data of parallel scientific Python codes on Piz Daint is possible.

%}}}

%{{{ cray perftools
%   ####   #####     ##     #   #
%  #    #  #    #   #  #     # #
%  #       #    #  #    #     #
%  #       #####   ######     #
%  #    #  #   #   #    #     #
%   ####   #    #  #    #     #
%% \section{Cray perftools}
%% 
%% CrayPAT (Cray Performance Measurement and Analysis toolset) is the recommended
%% performance analysis tool from Cray. CrayPAT consists of the following main
%% components:
%% 
%% \begin{itemize}
%% \setlength\itemsep{0.1em}
%% 	\item pat\_run - a simplified, easy-to-use version of CrayPAT for dynamically-linked executables,
%% 	\item perftools-lite - an easy-to-use version of CrayPAT,
%% 	\item perftools and PAT\_RT\_PERFCTR - CrayPAT and counters, for advanced users:
%%     \begin{itemize}
%%     \setlength\itemsep{0em}
%% 	\item pat\_build - used to instrument the program to be analyzed,
%% 	\item pat\_report - a standalone text report generator that can be use to further explore the data generated by instrumented program execution,
%%     \end{itemize}
%% 	\item Apprentice2 and pat\_view - graphical analysis tools that can be used, in addition to pat\_report to further explore and visualize the data generated by instrumented program execution.
%% 	\item Reveal and perftools-lite-loops - help to identify top time consuming loops, with compiler feedback on vectorization and recommendation for inserting OpenMP directives.
%% \end{itemize}
%% 
%% Better reframe checks and documentation would be required.
%% Memory analysis (perftools-lite-hbm) needs testing.
%% The support of CrayPAT in \colorbox{mycolor}{SHASTA} is unclear and should be verified.
%}}}

%{{{ blabla

%\begin{tabular}{|l|}
%   \hline
%   \hline
%\end{tabular}

%=========================================
% \begin{titlepage}
% 		\centering{
% 			{\fontsize{40}{48}\selectfont 
% 			A book title}
% 		}\\
% 			
% 		\vspace{10mm}
% 		\centering{\Large{Author Name}}\\
% 		\vspace{\fill}
% 		\centering \large{2011}
% \end{titlepage}


%=========================================
% \newpage{}
% \thispagestyle {empty}
% 
% \vspace*{2cm}
% 
% \begin{center}
% 	\Large{\parbox{10cm}{
% 		\begin{raggedright}
% 		{\Large 
% 			\textit{Do what you think is interesting, 
% 			do something that you think is fun and worthwhile, 
% 			because otherwise you won’t do it well anyway.}
% 		}
% 	
% 		\vspace{.5cm}\hfill{---Brian W. Kernighan}
% 		\end{raggedright}
% 	}
% }
% \end{center}
% }}}

\newpage
%=========================================
%\blinddocument

%=========================================
\nocite{*}
\printbibliography

%=========================================
%\begin{comment}
%Just some notes, not visible in pdf.
%\end{comment}


\end{document}
