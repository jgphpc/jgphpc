% CSCS course
\documentclass[a4paper,pagesize,12pt]{scrbook}

\usepackage{verbatim} % for comments
\usepackage{listings} 
\usepackage{xcolor}
\definecolor{mycolor}{rgb}{0.86666,0.78431,0.78431}
\usepackage{url}
% \usepackage{blindtext}
\newcommand{\q}[1]{>>\textit{#1}<<}
%\title{A book title}   
%\author{Author Name} 
%\date{\today} 

\begin{document}
\section*{Programming Environment for Practicals}
Your course account is: XXXX, your password is: YYYY

\subsection*{Accessing CSCS systems }
\subsubsection*{The front end node Ela}
In order to access CSCS systems, you need to login first to the front-end machine Ela which is accessible with: \colorbox{mycolor}{ssh -Y XXXX@ela.cscs.ch}
% \begin{itemize} \item \colorbox{mycolor}{ssh -Y XXXX@ela.cscs.ch} \end{itemize}

\subsubsection*{Piz Daint}
The supercomputer that we will use for all practicals is Piz Daint, a Cray XC30:\\
\url{ http://user.cscs.ch/computing_resources/piz_daint/index.html}\\
Piz Daint is accessible as: \colorbox{mycolor}{ssh -Y daint.cscs.ch}

Piz Daint is the main GPU/CPU hybrid supercomputing system with high scalability at CSCS, designed to run data parallel and computationally intensive applications.
Each compute node is equipped with an 8-core 64-bit Intel SandyBridge CPU (Intel Xeon E5-2670, 32 GB of memory) plus an NVIDIA Kepler K20X GPU with 6 GB GDDR5 memory. The nodes are connected by the "Aries" proprietary interconnect from Cray, with a dragonfly network topology. Piz Daint has 5272 compute nodes, corresponding to 42,176 CPU cores in total and 5272 GPUs. The peak performance is 7.8 Petaflops. 

\subsection*{Programming Environment}
After login into Piz Daint, the default programming environment is loaded. This environment provides convenient compilation wrappers based on the \emph{module} command to ensure that required include and libraries files are included and linked automatically. 
{\footnotesize \begin{itemize} 
        \item The command \colorbox{mycolor}{module list} will show your currently loaded modules.
        \item The command \colorbox{mycolor}{module avail} will show all the available modules. 
        \item To load a specific module, use \colorbox{mycolor}{module load mymodulename}, for example \emph{module load cray-hdf5}.
        \item To swap modules, use \colorbox{mycolor}{module swap modulenamecurrent modulenamedesired}.
        \item To see what the module command do, use \colorbox{mycolor}{module help modulename} or \colorbox{mycolor}{module show modulename}.
\end{itemize} }

\subsubsection*{Specific Compilation Environment}
In order to compile your code, you will need to select a compiler for your programming environment. Available compilers on Piz Daint are the Cray (default), Intel, Pgi, and Gnu compilers and these are loaded using: 

{\footnotesize \begin{itemize} 
\item \colorbox{mycolor}{module load PrgEnv-cray} for the CRAY compiler,
\item \colorbox{mycolor}{module swap PrgEnv-cray PrgEnv-intel} for the INTEL compiler,
\item \colorbox{mycolor}{module swap PrgEnv-cray PrgEnv-pgi} for the PGI compiler,
\item \colorbox{mycolor}{module swap PrgEnv-cray PrgEnv-gnu} for the GNU compiler.
\end{itemize} }

then use the wrapper scripts:
{\footnotesize \begin{itemize} 
\item \colorbox{mycolor}{ftn} for compiling Fortran code,
\item \colorbox{mycolor}{cc} for compiling C code,
\item \colorbox{mycolor}{CC} for compiling C++ code.
\end{itemize} }


\subsubsection*{Compiling a GPU code}
{\footnotesize \begin{itemize} 
\item \colorbox{mycolor}{module load craype-accel-nvidia35 } in addition to the Cray, Intel and Gnu compilers, 
\item \colorbox{mycolor}{module load craype-accel-nvidia35 ; module rm [cray-]libsci\_acc} in addition to the Pgi compiler.
\end{itemize} }

% To compile CUDA code use nvcc.

\subsection*{Launching jobs}
A set of compute nodes has been reserved for this course. For simple practicals, 1 node will be sufficient. The following command will give you 1 node (8 cores and 1 K20x gpu) for up to 15 minutes using the \emph{course} reservation:
{\footnotesize \begin{itemize} 
\item \colorbox{mycolor}{salloc $--$res=course -N1 -t15}
\end{itemize} }
You are now able to launch your jobs on the compute nodes. During the practicals, you will also use \emph{sbatch}.

\subsubsection*{Running MPI Jobs}
After starting \emph{salloc}, you can launch your jobs on the compute node with the \emph{aprun} command and the \emph{-n} flag to specify the number of MPI tasks.
This command executes 8 tasks of the mympiexe executable: \colorbox{mycolor}{aprun -n8 ./mympiexe}

\subsubsection*{Running OpenMP jobs}
Use the \emph{-d} aprun flag to specify the number of OpenMP threads. This command executes 8 threads of the myompexe executable:
{\footnotesize \begin{itemize} 
\item \colorbox{mycolor}{export OMP\_NUM\_THREADS=8}
\item \colorbox{mycolor}{aprun -n1 -d\$OMP\_NUM\_THREADS ./myompexe}
\end{itemize} }

\subsubsection*{Running MPI/OpenMP jobs}
This command executes a total of 8 processes (2 MPI tasks, each task running 4 threads) of the mympiompexe executable:
\colorbox{mycolor}{aprun -n2 -d4 ./mympiompexe}.\\

\subsubsection*{Exiting}
When you are done, end your VNC session with: \colorbox{mycolor}{vncserver -kill :1}.
When you are done, end your SSH session with: \colorbox{mycolor}{exit}.


%\begin{tabular}{|l|}
%   \hline
%   \hline
%\end{tabular}

%=========================================
% \begin{titlepage}
% 		\centering{
% 			{\fontsize{40}{48}\selectfont 
% 			A book title}
% 		}\\
% 			
% 		\vspace{10mm}
% 		\centering{\Large{Author Name}}\\
% 		\vspace{\fill}
% 		\centering \large{2011}
% \end{titlepage}


%=========================================
% \newpage{}
% \thispagestyle {empty}
% 
% \vspace*{2cm}
% 
% \begin{center}
% 	\Large{\parbox{10cm}{
% 		\begin{raggedright}
% 		{\Large 
% 			\textit{Do what you think is interesting, 
% 			do something that you think is fun and worthwhile, 
% 			because otherwise you won’t do it well anyway.}
% 		}
% 	
% 		\vspace{.5cm}\hfill{---Brian W. Kernighan}
% 		\end{raggedright}
% 	}
% }
% \end{center}

\newpage
%=========================================
%\blinddocument


%=========================================
%\begin{comment}
%Just some notes, not visible in pdf.
%\end{comment}


\end{document}
